\documentclass[]{article}
\usepackage{lmodern}
\usepackage{amssymb,amsmath}
\usepackage{ifxetex,ifluatex}
\usepackage{fixltx2e} % provides \textsubscript
\ifnum 0\ifxetex 1\fi\ifluatex 1\fi=0 % if pdftex
  \usepackage[T1]{fontenc}
  \usepackage[utf8]{inputenc}
\else % if luatex or xelatex
  \ifxetex
    \usepackage{mathspec}
  \else
    \usepackage{fontspec}
  \fi
  \defaultfontfeatures{Ligatures=TeX,Scale=MatchLowercase}
\fi
% use upquote if available, for straight quotes in verbatim environments
\IfFileExists{upquote.sty}{\usepackage{upquote}}{}
% use microtype if available
\IfFileExists{microtype.sty}{%
\usepackage{microtype}
\UseMicrotypeSet[protrusion]{basicmath} % disable protrusion for tt fonts
}{}
\usepackage[margin=1in]{geometry}
\usepackage{hyperref}
\hypersetup{unicode=true,
            pdftitle={Genetic drift},
            pdfborder={0 0 0},
            breaklinks=true}
\urlstyle{same}  % don't use monospace font for urls
\usepackage{color}
\usepackage{fancyvrb}
\newcommand{\VerbBar}{|}
\newcommand{\VERB}{\Verb[commandchars=\\\{\}]}
\DefineVerbatimEnvironment{Highlighting}{Verbatim}{commandchars=\\\{\}}
% Add ',fontsize=\small' for more characters per line
\usepackage{framed}
\definecolor{shadecolor}{RGB}{248,248,248}
\newenvironment{Shaded}{\begin{snugshade}}{\end{snugshade}}
\newcommand{\KeywordTok}[1]{\textcolor[rgb]{0.13,0.29,0.53}{\textbf{#1}}}
\newcommand{\DataTypeTok}[1]{\textcolor[rgb]{0.13,0.29,0.53}{#1}}
\newcommand{\DecValTok}[1]{\textcolor[rgb]{0.00,0.00,0.81}{#1}}
\newcommand{\BaseNTok}[1]{\textcolor[rgb]{0.00,0.00,0.81}{#1}}
\newcommand{\FloatTok}[1]{\textcolor[rgb]{0.00,0.00,0.81}{#1}}
\newcommand{\ConstantTok}[1]{\textcolor[rgb]{0.00,0.00,0.00}{#1}}
\newcommand{\CharTok}[1]{\textcolor[rgb]{0.31,0.60,0.02}{#1}}
\newcommand{\SpecialCharTok}[1]{\textcolor[rgb]{0.00,0.00,0.00}{#1}}
\newcommand{\StringTok}[1]{\textcolor[rgb]{0.31,0.60,0.02}{#1}}
\newcommand{\VerbatimStringTok}[1]{\textcolor[rgb]{0.31,0.60,0.02}{#1}}
\newcommand{\SpecialStringTok}[1]{\textcolor[rgb]{0.31,0.60,0.02}{#1}}
\newcommand{\ImportTok}[1]{#1}
\newcommand{\CommentTok}[1]{\textcolor[rgb]{0.56,0.35,0.01}{\textit{#1}}}
\newcommand{\DocumentationTok}[1]{\textcolor[rgb]{0.56,0.35,0.01}{\textbf{\textit{#1}}}}
\newcommand{\AnnotationTok}[1]{\textcolor[rgb]{0.56,0.35,0.01}{\textbf{\textit{#1}}}}
\newcommand{\CommentVarTok}[1]{\textcolor[rgb]{0.56,0.35,0.01}{\textbf{\textit{#1}}}}
\newcommand{\OtherTok}[1]{\textcolor[rgb]{0.56,0.35,0.01}{#1}}
\newcommand{\FunctionTok}[1]{\textcolor[rgb]{0.00,0.00,0.00}{#1}}
\newcommand{\VariableTok}[1]{\textcolor[rgb]{0.00,0.00,0.00}{#1}}
\newcommand{\ControlFlowTok}[1]{\textcolor[rgb]{0.13,0.29,0.53}{\textbf{#1}}}
\newcommand{\OperatorTok}[1]{\textcolor[rgb]{0.81,0.36,0.00}{\textbf{#1}}}
\newcommand{\BuiltInTok}[1]{#1}
\newcommand{\ExtensionTok}[1]{#1}
\newcommand{\PreprocessorTok}[1]{\textcolor[rgb]{0.56,0.35,0.01}{\textit{#1}}}
\newcommand{\AttributeTok}[1]{\textcolor[rgb]{0.77,0.63,0.00}{#1}}
\newcommand{\RegionMarkerTok}[1]{#1}
\newcommand{\InformationTok}[1]{\textcolor[rgb]{0.56,0.35,0.01}{\textbf{\textit{#1}}}}
\newcommand{\WarningTok}[1]{\textcolor[rgb]{0.56,0.35,0.01}{\textbf{\textit{#1}}}}
\newcommand{\AlertTok}[1]{\textcolor[rgb]{0.94,0.16,0.16}{#1}}
\newcommand{\ErrorTok}[1]{\textcolor[rgb]{0.64,0.00,0.00}{\textbf{#1}}}
\newcommand{\NormalTok}[1]{#1}
\usepackage{graphicx,grffile}
\makeatletter
\def\maxwidth{\ifdim\Gin@nat@width>\linewidth\linewidth\else\Gin@nat@width\fi}
\def\maxheight{\ifdim\Gin@nat@height>\textheight\textheight\else\Gin@nat@height\fi}
\makeatother
% Scale images if necessary, so that they will not overflow the page
% margins by default, and it is still possible to overwrite the defaults
% using explicit options in \includegraphics[width, height, ...]{}
\setkeys{Gin}{width=\maxwidth,height=\maxheight,keepaspectratio}
\IfFileExists{parskip.sty}{%
\usepackage{parskip}
}{% else
\setlength{\parindent}{0pt}
\setlength{\parskip}{6pt plus 2pt minus 1pt}
}
\setlength{\emergencystretch}{3em}  % prevent overfull lines
\providecommand{\tightlist}{%
  \setlength{\itemsep}{0pt}\setlength{\parskip}{0pt}}
\setcounter{secnumdepth}{0}
% Redefines (sub)paragraphs to behave more like sections
\ifx\paragraph\undefined\else
\let\oldparagraph\paragraph
\renewcommand{\paragraph}[1]{\oldparagraph{#1}\mbox{}}
\fi
\ifx\subparagraph\undefined\else
\let\oldsubparagraph\subparagraph
\renewcommand{\subparagraph}[1]{\oldsubparagraph{#1}\mbox{}}
\fi

%%% Use protect on footnotes to avoid problems with footnotes in titles
\let\rmarkdownfootnote\footnote%
\def\footnote{\protect\rmarkdownfootnote}

%%% Change title format to be more compact
\usepackage{titling}

% Create subtitle command for use in maketitle
\newcommand{\subtitle}[1]{
  \posttitle{
    \begin{center}\large#1\end{center}
    }
}

\setlength{\droptitle}{-2em}

  \title{Genetic drift}
    \pretitle{\vspace{\droptitle}\centering\huge}
  \posttitle{\par}
    \author{true}
    \preauthor{\centering\large\emph}
  \postauthor{\par}
      \predate{\centering\large\emph}
  \postdate{\par}
    \date{October, 2018}


\begin{document}
\maketitle

{
\setcounter{tocdepth}{1}
\tableofcontents
}
\section{Some popular R packages for population
genetics}\label{some-popular-r-packages-for-population-genetics}

Population genetics in R

\url{http://grunwaldlab.github.io/Population_Genetics_in_R/}

PopGenome: An Efficient Swiss Army Knife for Population Genomic Analyses

\url{https://cran.r-project.org/web/packages/PopGenome/}

\section{Simple genetic drift
program}\label{simple-genetic-drift-program}

\subsubsection{Before starting, think what is genetic
drift?}\label{before-starting-think-what-is-genetic-drift}

\subsubsection{}\label{section}

\begin{Shaded}
\begin{Highlighting}[]
\CommentTok{#----- simulates drift}
\CommentTok{# N is population size and p is initial allele frequency}
\NormalTok{drift <-}\StringTok{ }\ControlFlowTok{function}\NormalTok{(N,p) \{}
\NormalTok{   g=}\KeywordTok{as.numeric}\NormalTok{()}
\NormalTok{   t=}\DecValTok{0}
   \ControlFlowTok{while}\NormalTok{(p}\OperatorTok{>}\DecValTok{0} \OperatorTok{&}\StringTok{ }\NormalTok{p}\OperatorTok{<}\DecValTok{1}\NormalTok{) \{}
\NormalTok{      genotypes <-}\StringTok{ }\KeywordTok{rbinom}\NormalTok{(N,}\DecValTok{1}\NormalTok{,p)}
\NormalTok{      p<-}\KeywordTok{mean}\NormalTok{(genotypes)}
\NormalTok{      t=t}\OperatorTok{+}\DecValTok{1}
      \CommentTok{# the last column of each row contains current allele frequency}
\NormalTok{      genotypes=}\KeywordTok{cbind}\NormalTok{(}\KeywordTok{t}\NormalTok{(genotypes),p)}
\NormalTok{      g=}\KeywordTok{rbind}\NormalTok{(g,genotypes)}
\NormalTok{   \}}
   \KeywordTok{return}\NormalTok{(g[,N}\OperatorTok{+}\DecValTok{1}\NormalTok{])}
\NormalTok{\}}
\end{Highlighting}
\end{Shaded}

EXERCISE: Guess what the program does and why

HINT: look for help with `help(command)' or `? command'

HELP in R: \url{https://www.r-project.org/help.html}

\subsection{Running the drift R
function}\label{running-the-drift-r-function}

\begin{Shaded}
\begin{Highlighting}[]
\CommentTok{# num of individuals}
\NormalTok{N=}\DecValTok{20}
\CommentTok{# initial allele frequency}
\NormalTok{p0=.}\DecValTok{2}
\NormalTok{f=}\KeywordTok{drift}\NormalTok{(N,p0)}
\end{Highlighting}
\end{Shaded}

\begin{Shaded}
\begin{Highlighting}[]
\CommentTok{# list of colors for plotting each replicate}
\NormalTok{color =}\StringTok{ }\NormalTok{grDevices}\OperatorTok{::}\KeywordTok{colors}\NormalTok{()[}\KeywordTok{grep}\NormalTok{(}\StringTok{'gr(a|e)y'}\NormalTok{, grDevices}\OperatorTok{::}\KeywordTok{colors}\NormalTok{(), }\DataTypeTok{invert =}\NormalTok{ T)]}
\KeywordTok{plot}\NormalTok{(f, }\DataTypeTok{type=}\StringTok{'l'}\NormalTok{, }\DataTypeTok{ylim=}\KeywordTok{c}\NormalTok{(}\DecValTok{0}\NormalTok{,}\DecValTok{1}\NormalTok{), }\DataTypeTok{xlim=}\KeywordTok{c}\NormalTok{(}\DecValTok{1}\NormalTok{,}\DecValTok{30}\NormalTok{), }\DataTypeTok{xlab=}\StringTok{'generation'}\NormalTok{, }\DataTypeTok{ylab=}\StringTok{'f'}\NormalTok{)}
\ControlFlowTok{for}\NormalTok{ (rep }\ControlFlowTok{in} \KeywordTok{seq}\NormalTok{(}\DecValTok{20}\NormalTok{)) \{}
\NormalTok{    f =}\StringTok{ }\KeywordTok{drift}\NormalTok{(N, p0)}
    \KeywordTok{lines}\NormalTok{(f, }\DataTypeTok{col=}\KeywordTok{sample}\NormalTok{(color,}\DecValTok{1}\NormalTok{))}
\NormalTok{\}}
\end{Highlighting}
\end{Shaded}

\includegraphics{drift_files/figure-latex/unnamed-chunk-3-1.pdf}

EXERCISE: try different sizes (N) an dinitial frequencies (p)

EXERCISE: What is the probability of fixation of allele 1 ?


\end{document}
